
-------------------------------------------------
Stage 1. The nature of alt-right rhetoric
-------------------------------------------------

** Stage 1 analysis at video-level

1.1. Argument: Moralistic arguments and rhetoric support a simplistic picture of the world (aka, 'simplism')
1.1.1. Implications and comparisons: More moralistic arguments are associated with more simplism
1.1.2. Implications and comparisons: Less moralistic arguments are associated with more sophistication / analytic thinking
1.1.3. Measurement:
1.1.4. Analytical strategy:
1.1.5. Open questions: currently, NA
1.1.6. Comments: currently, NA

1.2. Argument: A simplistic depiction of the world allows for alt-right to make a contradictory claim. Namely, an 'in-group' is both strong and powerful but, at the same time, a victim of outsiders (i.e., the 'out-group'). Note that this is a rhetorical move; the content (e.g., what the in-group is, how victimhood is experienced) is less relevant.
1.2.1. Implications and comparisons: Higher simplism is associated with more claims of being powerful yet also a victim
1.2.2. Implications and comparisons: Less simplism is associated with different claims regarding power and victimization
1.2.3. Measurement:
1.2.4. Analytical strategy:
1.2.5. Open questions: How to 'control' for potential confounding rhetorical strategies
1.2.6. Comments: Emphasize rhetoric (syntactic features?) rather than content (semantic features?)

1.3. Argument: After making claims of 'power-yet-victimization', alt-right advocate for using their power to exclude the victimizers from an imagined future version of society.
1.3.1. Implications and comparisons: Rhetoric of power-yet-victimization is associated with calls for exclusion
1.3.2. Implications and comparisons: Less rhetoric of power-yet-victimization is associated with fewer calls for exclusion
1.3.3. Measurement:
1.3.4. Analytical strategy:
1.3.5. Open questions: How to 'control' for potential confounding rhetorical strategies that could potential link to exclusion?
1.3.6. Comments: Emphasize rhetoric (syntactic features?) rather than content (semantic features?) 

1.4. Argument: After calling for exclusion, radicals would articulate strategies for creating exclusion, e.g., violence, conversion. However, alt-right activists leave the _method_ unsaid, and only implied.
1.4.1. Implications and comparisons: Compare an artificial baseline of calls for violence to what alt-right activists do (or do not) call for.
1.4.2. Implications and comparisons: Others...?
1.4.3. Measurement: 
1.4.4. Analytical strategy: Potentially use language models trained on the data and then manipulate the data to see what would have to be said to produce calls for violence? Manipulation could occur by adding phrases from Afghan text. So, we'd potentially see how much alt-right rhetoric would have to shift towards the 'Afghan radical' direction to become violent.
1.4.5. Open questions: 
1.4.6. Comments: 

-------------------------------------------------
Stage 2. The amplification of alt-right rhetoric
-------------------------------------------------

** Stage 2 analysis at video- and comment level

2.1. Argument: The videos and comments are related by content and/or rhetoric
DONE. We sample from the comments (n = 38,000) and collect the sampled comments' associated videos. Then we embed the comments and transcripts in Google's embedding space. Next, we compare the the mean embedding vector of transcripts to the mean embedding vector of comments using a normalized distance and the cosine similarity. We find that both videos and comments are more similar to non-hate speech on Stormfront than to non-hate and hate-speech on Twitter.



-------------------------------------------------
Stage 3. The social dynamics of amplification
-------------------------------------------------

** Stage 3 analysis at video-, comment- , and user- level